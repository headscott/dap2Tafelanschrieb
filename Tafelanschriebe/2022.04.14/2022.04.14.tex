\subsection{Quicksort nach Dijkstra in C}

\textbf{Live-Programmierung:}
Pipe-Operator: $|$(output des linken Teils wird als Input des rechtens gehandhabt)\\
\\
shuf mischt die Eingabe
\begin{center}
\begin{lstlisting}
seq 10 | shuf
\end{lstlisting}
\end{center}

\textbf{Tafel:}\\

\begin{tabular}{|c|c|c|c|c|c|c|c|c|c|}
    \hline
    1 & 2&3&4&5&6&7&8&9\\\hline
\end{tabular} $\overset{Input}{\rightarrow}$ \begin{tabular}{|c|}
    \hline
    A\\\hline
\end{tabular}\\

Wie bekommt man es hin, dass die der Speicher nicht zu groß wird aus der Eingabe heraus?\\
Erster Vorschlag:
\begin{tikzpicture}
\node (1) at (0, 0) [rectangle] {1};
\node (2) at (1, 0) [rectangle] {2};
\node (3) at (2, 0) [rectangle] {3};
\draw[->] (1) to (2);
\draw[->] (2) to (3);
\end{tikzpicture}
Zweiter Vorschlag:\\
Man erstellt einen Speicher und wenn der vom Platz nicht reicht wird, dieser einfach verdoppelt:
\begin{center}
    \begin{tabular}{|c|}
        \hline 1 \\\hline
    \end{tabular}
    \\
    \begin{tabular}{|c|c|}
        \hline 1 & 2 \\\hline
    \end{tabular}
    \\
    \begin{tabular}{|c|c|c|c|}
        \hline 1 & 2 & 3 & 4\\\hline
    \end{tabular}
\end{center}

\newpage
\textbf{Live-Programmierung:}\\
Programm für die Speicherallocation bei zu großen Eingaben
\begin{lstlisting}
/* Dieser Code wird ins Vorlesungs Git-Repo hochgeladen */
#include<stdlib.h>
#include<stdio.h>

char *input=NULL;
int length=0;

int read_input(void) {
    int size=0; /* size of memory block */
    int fill=0; /* number characters read */
    int c; /* next character */
    
    while((c=getchar()) != EOF){
        if(fill>=size){ /* wenn der Speicher nicht reicht */
            int i,newsize=2*size+1;
            char *newinput;
            
            /*Bestell Speicher in der entsprechenden Groesse */
            if((newinput=(char *)malloc(sizeof(char)*newsize))==NULL) 
                return -1;
            for(i=0;i<size;i++) newinput[i]=input[i];
            if(input!=NULL) free(input);
            input=newinput;
            size=newsize;
        }
        input[fill++]=(char)c; /*leg das neue Symbol ab */
    }
    length=fill;
    return length; /* Gibt neue Laenge zurueck */
}

int main(int argc, char **argv){
    int i;
    
    if(read_input()<=0) return 1;
    for(i=0;i<length;i++) putchar(input[i]);
}
\end{lstlisting}

Es muss jetzt noch die Eingabe in richtige Integer-Werte übersetzt werden. Dies haben wir an der Stelle leider nicht geschafft. Er bereitet dies zum nächsten Mal vor und zeigt uns anschließend das Ergebnis.