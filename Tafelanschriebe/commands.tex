%%%%%%%%%%%%%%%%%%%%%%%%%%%%%%%
%   Mathematische Ausdrücke   %
%%%%%%%%%%%%%%%%%%%%%%%%%%%%%%%

%%%Integrale
%Integral von 0 bis pi/2
\newcommand{\intpihalbe}{\int_{0}^{\frac{\pi}{2}}}
\newcommand{\intnullbisn}{\int_{0}^{n}}
\newcommand{\inteinsbisn}{\int_{1}^{n}}

%%%Summen
\newcommand{\summeeinsbisn}{\sum_{i = 1}^{n}}
\newcommand{\summeeinsbisnpluseins}{\sum_{i = 1}^{n+1}}
\newcommand{\summenullbisn}{\sum_{i = 0}^{n}}
\newcommand{\summenullbisnpluseins}{\sum_{i = 0}^{n+1}}

%%%Produkte
\newcommand{\produkteinsbisn}{\Pi_{i=1}^{n}}
\newcommand{\produkteinsbisnpluseins}{\Pi_{i=1}^{n+1}}
\newcommand{\produktnullbisn}{\Pi_{i=0}^{n}}
\newcommand{\produktnullbisnpluseins}{\Pi_{i=0}^{n+1}}

%%%Limes
\newcommand{\limesngegenunendlich}{\lim_{n \rightarrow \infty}}
\newcommand{\limesngegenminusunendlich}{\lim_{n \rightarrow - \infty}}
\newcommand{\limesxgegenunendlich}{\lim_{x \rightarrow \infty}}
\newcommand{\limesxgegenminusunendlich}{\lim_{x \rightarrow - \infty}}

%%%%%%%%%%%%%%%%%%%%%%%%%%%%%%%%%%%%%%
%   Umrandungen um Matheausdruecke   %
%%%%%%%%%%%%%%%%%%%%%%%%%%%%%%%%%%%%%%

%Schwarzer Kasten um Ausdruck => Es muss $formel$ angegeben werden
\newcommand*{\rectangled}[1]{\tikz[baseline=(char.base)]{
\node[shape=rectangle,draw,inner sep=2pt] (char){#1};}}
%Roter Kasten um Ausdruck => Es muss $formel$ angegeben werden
\newcommand*{\redrectangled}[1]{\tikz[baseline=(char.base)]{
\node[shape=rectangle,draw,inner sep=2pt, red] (char) {\textcolor{black}{#1}};}}
%Schwarzer Kreis um Ausdruck => Es muss $formel$ angegeben werden
\newcommand*\circled[1]{\tikz[baseline=(char.base)]{
\node[shape=circle,draw,inner sep=1pt] (char) {#1};}}
%Roter Kreis um Ausdruck => Es muss $formel$ angegeben werden
\newcommand*\redcircled[1]{\tikz[baseline=(char.base)]{
\node[shape=circle,draw,inner sep=1pt] (char) {\textcolor{red}{#1}};}}


%%%%%%%%%%%%%%%%%%
%   Strukturen   %
%%%%%%%%%%%%%%%%%%
%Erzeugt einen beispielhaften Binärbaum
\newcommand{\beispielBaum}{
    \node(root) at (0,0)  [draw, circle, inner sep=4]  {$0$};
    \node(1) at (-4,-1)   [draw, circle, inner sep=4]  {$1$};
    \node(2) at (4,-1)    [draw, circle, inner sep=4]  {$2$};
    %Knoten unter 1
    \node(11) at (-6,-2)  [draw, circle, inner sep=4]  {$4$};
    \node(12) at (-2,-2)  [draw, circle, inner sep=4]  {$5$};
    %Knoten unter 2
    \node(21) at (2,-2)   [draw, circle, inner sep=4]  {$6$};
    \node(22) at (6,-2)   [draw, circle, inner sep=4]  {$7$};
    %Kanten
    %root
    \draw[-] (1) to (0);
    \draw[-] (2) to (0);
    %Kanten nach 1
    \draw[-] (11) to (1);
    \draw[-] (12) to (1);
    %Kanten nach 2
    \draw[-] (21) to (2);
    \draw[-] (22) to (2);
}

%Erzeugt eine Liste liste{anzahlElemente}{Element1 & Element2 & ...}
\newcommand{\liste}[2]{
{\renewcommand{\arraystretch}{1.5}
\begin{tabular}{*#1{|c}|}
    \hline
    #2\\\hline
\end{tabular}}
}